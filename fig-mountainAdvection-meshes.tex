\documentclass{article}
\usepackage[paperwidth=6.1in,paperheight=3.8in,margin=0in]{geometry}
\usepackage{mathptmx}%{times}
\usepackage{graphicx}
\usepackage{siunitx}
\usepackage[absolute,overlay]{textpos}
\usepackage{color}

\setlength{\parindent}{0pt}
\begin{document}
\TPMargin{2pt}
\textblockcolour{white}
\begin{textblock}{2}(2,0.5)
\normalsize
Cut cells
\end{textblock}
\begin{textblock}{2.5}(2,6.5)
\normalsize
Slanted cells
\end{textblock}
\begin{center}
\includegraphics{../mountainAdvection-cutCell-5000-linearUpwind-1000/constant/meshW.pdf} \\
\hspace*{0.8em}\includegraphics{../mountainAdvection-slantedCell-5000-linearUpwind-1000/constant/mesh.pdf}
\end{center}
Two-dimensional meshes on a $x$--$z$ plane.  An idealised wave-shaped mountain with the standard cut cell method and the new slanted cell method.  The cut cell mesh has extremely small cells that can severely constrain the time-step for explicit methods, but the slanted cell mesh has cells that are long in the direction parallel to the lower boundary.
\end{document}
